
% This LaTeX was auto-generated from an M-file by MATLAB.
% To make changes, update the M-file and republish this document.

\documentclass{article}
\usepackage{graphicx}
\usepackage{color}

\sloppy
\definecolor{lightgray}{gray}{0.5}
\setlength{\parindent}{0pt}

\begin{document}

    
    
\subsection*{Contents}

\begin{itemize}
\setlength{\itemsep}{-1ex}
   \item ejemplo de inversion en el tiempo de los campos de desplazamiento de los
   \item dominio de la inversión
   \item visualizacion de la inversion de los datos
\end{itemize}


\subsection*{ejemplo de inversion en el tiempo de los campos de desplazamiento de los}

\begin{par}
La función import events transforma los archivos generados con script en python a una lista de objetos del tipo Event
\end{par} \vspace{1em}
\begin{verbatim}
%ev = importEvents();
\end{verbatim}
\begin{par}
con los archivos ya importados en la variable ev = $ev_1,\cdot, ev_n$ donde $$n$\$ es la cantidad de geosensores. tomamos un evento sísmico de los n existentes, podemos considerar por ejemplo, el evento $3$
\end{par} \vspace{1em}
\begin{verbatim}
index = 3;
% se invierte el campo de desplazamiento remuestreado.
%[X, Y, Z, X_domain, Y_domain, Z_domain, T_domain] = reverse_signal(ev(index));
\end{verbatim}


\subsection*{dominio de la inversión}

\begin{par}
el dominio definido por X\_domain, Y\_domain, Z\_domain, T\_domain son de $$60 \times 60 \times 10 \times 50$\$ cuya presición de la discretización está dada por $$dx =  14.7890$\$, $$dy =  13.9790$\$, $$dz = 34.7967$\$ metros y dependen de la cobertura de los geófonos, este cubo es en realidad el cubo de volumen mínimo que contiene a todos los geófonos.
\end{par} \vspace{1em}


\subsection*{visualizacion de la inversion de los datos}

\begin{par}
con las mediciones ya obtenidas de la unversión de las señales se considera la visualización de la norma del campo vectorial para visualización.
\end{par} \vspace{1em}
\begin{verbatim}
N = sqrt(X.^2 + Y.^2 + Z.^2);
\end{verbatim}



\end{document}
    
